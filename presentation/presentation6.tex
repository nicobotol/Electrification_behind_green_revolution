\documentclass[aspectratio=169, 12pt]{beamer}
% \documentclass[aspectratio=169, handout]{beamer} % for handout
\usepackage{pgfpages}
\usepackage[english]{babel}
\usepackage{booktabs,listings}
\usepackage[T1]{fontenc}
\usepackage[utf8]{inputenc}
\usepackage{xcolor}
\usepackage{graphicx}
\usepackage{graphics}
\usepackage{epstopdf}
\usepackage{upgreek}
\usepackage{amsmath}
\usepackage{textcomp}
\usepackage{booktabs}
\usepackage{tikz}
\usetikzlibrary{positioning,shapes,arrows,calc}
\usepackage{adjustbox}
\usepackage{siunitx}
\newcommand{\mesunt}[1]{\left[\si{#1}\right]}
\usepackage[backend=biber,style=numeric,sorting=none,citestyle=authortitle]{biblatex}
\addbibresource{../reference_electrification/biblio.bib}


\lstset{basicstyle=\ttfamily}
\setlength{\parskip}{.5\baselineskip}
\usetheme[style=vertical, frametotal=true]{NTNU}

\mode<handout>{%
    \pgfpagesuselayout{2 on 1}[a4paper] 
    % \setbeameroption{show notes}
}

\AtBeginSection[]
{
  \begin{frame}
    \frametitle{Table of Contents}
    \tableofcontents[currentsection]
  \end{frame}
}

\AtBeginSubsection[]
{
  \begin{frame}
    \frametitle{Table of Contents}
    \tableofcontents[currentsubsection]
  \end{frame}
}

\title{Floating PV and measure of the lost of inertia }
\subtitle{The electrification behind the green revolution}
\author{Andreetta Niccolò}
\date{July 2024}

\begin{document}
  \maketitle

  \begin{frame}[fragile]{Outline}
    \tableofcontents
  \end{frame}

%   ______     __
%  |  _ \ \   / /
%  | |_) \ \ / / 
%  |  __/ \ V /  
%  |_|     \_/   
               

\section{Floating solar PV}
\subsection{Motivation and benefits}
\begin{frame}{PV current trend}{\insertsubsection}
\begin{columns}
  \begin{column}{0.6\columnwidth}
    Renewable electricity capacity additions
    \begin{figure}
    \centering
    \includegraphics[width=\columnwidth]{figure/rec_shares.png}
  \end{figure}
\end{column}
\begin{column}{0.4\columnwidth}
  Cumulative PV power capacity
  \begin{figure}
    \centering
    \includegraphics[width=\columnwidth]{figure/pv_installation.png}
    \end{figure}
  \end{column}
\end{columns}

  {\tiny Source: \cite{irena2019}, \cite{iea2023}}
\end{frame}

\begin{frame}{PV plants classification}{\insertsubsection}
  Traditionally we may identify two families of PV plants

  \textcolor{NTNUBlue}{Utility scale PV}: installed on dedicated locations, such as agricultural land. Their installed capacity may exceed 5 MW.

  \textcolor{NTNUBlue}{Distributed PV}: installed on existing structures, in the proximity of users. The installed capacity is between $\le$ 10 kW and 100 kW.

\end{frame}

\begin{frame}{PV plants classification: pros and cons}{\insertsubsection}
  \begin{columns}
    \begin{column}{0.5\columnwidth}
      {\center \textcolor{NTNUBlue}{Utility scale}}\\
      \textcolor{NTNUgreen}{\textbf{+}} Cost effective: \$/MW half of rooftop installations\\
      \textcolor{NTNUgreen}{\textbf{+}} Extended life and reduced maintenance costs\\
      \textcolor{red}{\textbf{-}} Use of large land area\\
      \textcolor{red}{\textbf{-}} Degradation of wildlife habitat
    \end{column}
    \begin{column}{0.5\columnwidth}
      {\center \textcolor{NTNUBlue}{Distributed}}\\
      \textcolor{NTNUgreen}{\textbf{+}} Electricity is primary consumed by system owner, leading to lower cost for them\\
      \textcolor{NTNUgreen}{\textbf{+}} Reduced line losses\\
      \textcolor{red}{\textbf{-}} Limited space
    \end{column}
  \end{columns}
\end{frame}

\begin{frame}{Offshore PV (OPV)}{\insertsubsection}
  Solve the problem of large scale PV system deployment without reducing land development. \\
  Furthermore: 
  \begin{itemize}
    \item reduce water evaporation
    \item water reduces temperature $\Rightarrow$ increase efficiency
    \item integration with Offshore Wind Turbine
  \end{itemize} 

  Nowadays the share of OPV is small, but the technology is relative young (first installation in 2007)
\end{frame}

\subsection{Mooring systems}
\begin{frame}{Principal solutions}{\insertsubsection}
  How can we ensure that the plant remains in a fixed position and is not tow away?

  Principal solutions:\\
  \textcolor{NTNUBlue}{MONOPILE} or \textcolor{NTNUBlue}{FLOATING}
  
\end{frame}

\begin{frame}{Monopile}{\insertsubsection}
  Suitable for shallow water depth less than 5 m (e.g. aquaculture ponds)\\
  In China projects for 1 GW installed capacity (8 km from shore)

  \begin{figure}
    \centering
    \includegraphics[width=0.3\columnwidth]{figure/floating_PV_china.png}
  \end{figure}

  \textcolor{NTNUgreen}{\textbf{+}} The mooring system offers stability\\
  \textcolor{NTNUgreen}{\textbf{+}} Reduction of water surface temp. $\Rightarrow$ Good for fish farming\\
  \textcolor{red}{\textbf{-}} Marine organism may attach and faster the pile corrosion\\
  \textcolor{red}{\textbf{-}} Need of transport of the piles

  {\tiny Figure source: \cite{jmse11112064}}
\end{frame}

\begin{frame}{Floating}{\insertsubsection}
  \textcolor{NTNUBlue}{Idea:} The PV is mounted on a floating module whom movement is restricted through a multi-point mooring system 
  
  \textcolor{NTNUBlue}{Relatively young technology: }First installation in 2007

\end{frame}

\subsection{Types of floating structures}
\begin{frame}{Pontoon Floaters Module}{\insertsubsection}
  \begin{figure}
    \centering
    \includegraphics[width=0.4\columnwidth]{figure/pontoon.png}
  \end{figure}
  
  \textcolor{NTNUgreen}{\textbf{+}} Pontoons made of HDPE, which is resistant and recyclable\\
  \textcolor{NTNUgreen}{\textbf{+}} Easy installation\\
  \textcolor{NTNUgreen}{\textbf{+}} Covering water surface reduce evaporation\\
  \textcolor{NTNUgreen}{\textbf{+}} Cooling with water circulation and wind gusts effects\\
  \textcolor{red}{\textbf{-}} Interaction between floating bodies difficult to be modelled\\
  \textcolor{red}{\textbf{-}} Aquatic organism may attach the floating body and sink it

  {\tiny Figure source: \cite{jmse11112064}}
\end{frame}

\begin{frame}{Very Large Floating Structure}{\insertsubsection}
  Floating module consists of a floating body and a flexible film on which PV are laid on.
  \begin{figure}
    \centering
    \includegraphics[width=0.3\columnwidth]{figure/vlfs.png}
  \end{figure}

  \textcolor{NTNUgreen}{\textbf{+}} Smaller air gap with water and better cooling effect\\
  \textcolor{NTNUgreen}{\textbf{+}} Flexible structure better accomplish wave action and so more suitable for offshore deployment\\
  \textcolor{red}{\textbf{-}} Single module is bulky and so difficult to install\\
  % \textcolor{red}{\textbf{-}} Deep sea installation requires special mooring

  {\tiny Figure source: \cite{jmse11112064}}
\end{frame}

\begin{frame}{Very flexible floating structures (VFFS)}{\insertsubsection}
  PV module are directly placed on the water's surface forming long structures (which can withstand large deflections).

  \begin{figure}
    \centering
    \includegraphics[width=0.3\columnwidth]{figure/vffs.png}
  \end{figure}

  \textcolor{NTNUgreen}{\textbf{+}} Excellent water-cooling effect\\
  \textcolor{NTNUgreen}{\textbf{+}} Reduced weight and simple mooring system\\
  \textcolor{NTNUgreen}{\textbf{+}} Reduce installation costs, possibility to tow ashore during winter\\
  \textcolor{red}{\textbf{-}} Earlier development stage

  {\tiny Figure source \cite{vffs_figure}}
\end{frame}

\subsection{Installation example}
\begin{frame}{Example: EDP Renewables APAC in Singapore}
  Plant located in the Straits of Johor (Between Singapore and Malaysia)
  \begin{columns}
    \begin{column}{0.62\columnwidth}
      \begin{figure}
        \centering
        \includegraphics[width=\columnwidth]{figure/floating_edpr.jpg}
      \end{figure}
    \end{column}
    \begin{column}{0.38\columnwidth}
      \begin{table}
        \begin{tabular}{ll}
          Installation & March 2021\\
          Water depth & $\backsim $ 12 m\\
          Type & Nearshore \\
          Capacity & 5 MWp\\
          Surface & 50 000 \si{\square\meter}\\
          Panels & 13 312 \\
          Inverters & 40 \\
          Floaters & 30 000\\
          Energy & 6 GWh\\
        \end{tabular}
      \end{table}
    \end{column}
  \end{columns}
  {\tiny Source: \cite{edpr_floating_solar}, \cite{straitstimes_floating_solar_farm}}
\end{frame}

\subsection{Summary}
\begin{frame}{Summary}
  \textcolor{NTNUBlue}{Environmental impact}: less consume of soil; reduce evaporation; impact of wildlife

  \textcolor{NTNUBlue}{PV module}: development of the semiconductor technologies and improved efficiency for the lower temperature; necessity of lighter structures for floating
  
  \textcolor{NTNUBlue}{Floating structure}: up to now mainly use of rigid structures, but nowadays also lightweight composite materials start to be used

  \textcolor{NTNUBlue}{Mooring system}: required increasing resistance for withstand the loads. Anti corrosion and anti biological parasites feature are important

  \textcolor{NTNUBlue}{Transportation and installation}: easier than with the use of monopiles

\end{frame}

%   __  __      _        _          
%  |  \/  | ___| |_ _ __(_) ___ ___ 
%  | |\/| |/ _ \ __| '__| |/ __/ __|
%  | |  | |  __/ |_| |  | | (__\__ \
%  |_|  |_|\___|\__|_|  |_|\___|___/
                                  

\section{Quantification of the lost of inertia in the power system}
\begin{frame}{\insertsection}
  Increase of energy from renewable energy resources implies lost of inertia in the grid.
  \begin{figure}
    \centering
    \includegraphics[width=0.5\columnwidth]{figure/frequency_disturbance.png}
  \end{figure} 
\end{frame}

\subsection{Metrics}
\begin{frame}{Time domain of the post-fault frequency evolution}{Classical metrics}
  \begin{itemize}[<+(1)->]
    \item Rate of Change of Frequency
    \begin{equation}
      \lvert \dot{\omega} \rvert _{max} = \max_i\left(\max_{t\ge0}\lvert \dot{\omega}_i(t) \rvert\right)
    \end{equation}
    
    \item Frequency Nadir
    \begin{equation}
      \lvert \underline{\omega} \rvert = \max_i\lvert\min_{t\ge0} \omega_i(t) \rvert
    \end{equation}
    \item Peak virtual inertia power injection
    \begin{equation}
      \bar{p}_v = \max_i\left(\max_{t\ge0}\lvert p_{v,i}(t) \rvert\right)
    \end{equation}
  \end{itemize}
\end{frame}

\begin{frame}{Damping and inertia}{Classical metrics}
  Eigenvalue analysis
  \begin{itemize}
    \item Damping ratio of the power system
    \begin{equation}
      \zeta_{min} = \min_{k}\frac{-\sigma_k}{\sqrt{\sigma_k^2 + \omega_k^2}}
    \end{equation}
    with $\lambda_k=\sigma_k + \text{i}\omega_k$ the $k-th$ eigenvalue
  \end{itemize}
  
  Physical and virtual inertia of the devices
\begin{itemize}
  \item Total inertia
  \begin{equation}
    H_{total} = \sum_{i}H_i + \sum_{i} \tilde{H}_i = \sum_{i}\frac{m_i \omega_0}{2 S_{rated,i}} + \sum_{i} \frac{\tilde{m}_i \omega_0}{2 S_{rated,i}}
  \end{equation} 
\end{itemize}
\end{frame}

\begin{frame}{Energy metrics}{}
  \begin{itemize}[<+(1)->]
      \item Total energy imbalance
      \begin{equation}
        E_{\tau,\omega}=\int_{0}^{\tau}\sum_{i} q_i\omega_i^{2}dt=\int_{0}^{\tau} \omega^T Q \omega dt
      \end{equation}
      
      \item Total virtual inertia effort
      \begin{equation}
        E_{\tau,m}=\int_{0}^{\tau}\sum_{i} r_{m,i}p_{m,i}^{2}dt=\int_{0}^{\tau} p_{m}^T R_m p_{m} dt
      \end{equation}

      \item Total virtual damping effort
      \begin{equation}
        E_{\tau,d}=\int_{0}^{\tau}\sum_{i} r_{d,i}p_{d,i}^{2}dt=\int_{0}^{\tau} p_{d}^T R_d p_{d} dt
      \end{equation}
    \end{itemize}

    $q_i, r_m, and r_d$ are weights used for penalize one effort or the other
\end{frame}

\begin{frame}{$\mathcal{H}_2$ norm}
  The energy of the response to an impulse fault \textit{or} expected energy of the response to a white noise
    \begin{equation}
      \int_{0}^{\infty}\| y_p \|^2 dt = \int_{0}^{\infty} \omega^T Q \omega + p_{m}^T R_m p_{m} + p_{d}^T R_d p_{d} dt
    \end{equation}
    \textcolor{NTNUViolet}{
    Example: 
    \begin{gather}
      \text{System } \mathcal{G} :
      \begin{cases}
        \dot{x} = A x + G \eta \\
        y_p = C x
      \end{cases}
      \Rightarrow \| \mathcal{G} \|_2^2=\text{trace}\left(G^T P G\right)\\
      \text{Observability gramian: } PA + A^TP+C^TC=0
    \end{gather} 
    }
\end{frame}

\begin{frame}{$\mathcal{H}_{\infty }$ norm}
  
  $\mathcal{H}_{\infty }$ norm: intuitively, it gives the maximum amplification of an input signal at the output that is caused by the system over all possible input signals

  Defined as the supremum of the largest singular number of its transfer function over the imaginary axis:
  \begin{equation}
    \| \mathcal{H} \|_{\infty}= \sup_{\omega \in \mathbb{R} } \sigma_{max}\left(\mathcal{G}\left(j\omega\right)\right)
  \end{equation}
  
\end{frame}

\subsection{Example}
\begin{frame}{Use of the metrics}
  Test on the Kundor model with two different damping configuration
  \begin{columns}
    \begin{column}{0.5\columnwidth}
      \includegraphics[width = \columnwidth]{figure/kundor_frequency.png}
    \end{column}
    \begin{column}{0.5\columnwidth}
      \begin{table}[h!]
        \centering
        \begin{tabular}{lcc}
        \toprule
        & Allocation 1 & Allocation 2 \\
        \midrule
        $H_{\text{total}}$ & 40.85 s & 40.85 s \\
        $\zeta_{\text{min}}$ & 0.1190 & 0.1206 \\
        $\left|\dot{\omega}\right|_{\max}$ & 0.8149 Hz/s & 0.8135 Hz/s \\
        $\left|\omega\right|$ & 84.8 mHz & 65.1 mHz \\
        $\mathcal{H}_2 $ & 1.5337 & 0.6522 \\
        $\mathcal{H}_{\infty} $ & 0.7454 & 0.2782 \\
        $\overline{P}_v$ & 118.38 MW & 7.0446 MW \\
        \bottomrule
        \end{tabular}
      
        \label{table:allocations}
        \end{table}
    \end{column}
  \end{columns}

  {\tiny Source: \cite{Increasing_the_Resilience}}
\end{frame}

% \section{PLL}  
% \begin{frame}{PLL schema}{\insertsection}
%   \begin{figure}
%     \input{scheme/PLL2.tex} 
%   \end{figure}
% \begin{itemize}
%   \item Phase Detector \textcolor{NTNUBlue}{PD}
%   \item Loop Filter \textcolor{NTNUBlue}{LP}
%   \item Voltage Controlled Oscillator \textcolor{NTNUBlue}{VCO}
% \end{itemize}
% \end{frame}

% \begin{frame}{Phase Detector}{\insertsection}
%   \begin{figure}
%     \input{scheme/PD.tex} 
%   \end{figure}
%   Measures the $v_{abc}(t)$ voltage and \textcolor{NTNUBlue}{transforms} it in a $dq$ reference frame.
%   \begin{equation*}
%     abc \Rightarrow \alpha\beta \Rightarrow dq
%   \end{equation*}

%   Extracts the $v_q(t)$ component
% \end{frame}

% \begin{frame}{Loop Filter}{\insertsection}
%   \begin{figure}
%     \input{scheme/LF.tex} 
%   \end{figure}
%  Takes the error between the measured voltage $v_q$ and the estimated one $\hat{v}_q$ and computes the estimation of the frequency deviation at the bus connection $\Delta \hat{\omega}$
% \end{frame}

% \begin{frame}{Voltage-Controlled Oscillator}{\insertsection}
%   \begin{figure}
%     \tikzstyle{block} = [draw, fill=white!100, rounded corners]
\tikzstyle{block_red} = [draw, color=red, fill=white!100, rounded corners]
\tikzstyle{sum_plus_nw} = [draw, fill=white, circle, label=above left:$+$]% sum with + in north west position
\tikzstyle{sum_minus_se} = [draw, fill=white, circle, label=below left:$-$]
\tikzstyle{input} = [coordinate]
\tikzstyle{output} = [coordinate]

\adjustbox{width=0.75\columnwidth, center}{
\begin{tikzpicture}[auto, >=latex']  
  \node [style = input] (v_abc) { $v_{abc}$}; % delta P input
  \node [style = block, right of = v_abc, node distance = 1.2cm] (PD) {PD};
  \node [style = sum_minus_se, right of = PD] (sum_P) {}; 

  \node [style = block, right of = sum_P] (LF) {LF};
  \node [style = block_red, right of = LF, node distance = 1.7cm] (VCO) {VCO};

  \node [style = output, right of = VCO] (fake_point) {};
  \node [style = output, below of = fake_point, node distance = 0.7cm] (fake_point2) {};
  \node [style = output, right of = fake_point, node distance = 0.5cm] (output) {};
  
  \draw [->] (v_abc) -> node[above] {$v_{abc}$} (PD);
  \draw [->] (PD) -> node[above] {$v_{q}$}  (sum_P);
  \draw [->] (sum_P) -> node[above] {$\epsilon_q$} (LF);
  \draw [->] (LF) -- node {$\Delta \hat{\omega}$} (VCO);
  \draw [-] (VCO) -- node {} (fake_point);
  \draw [->] (fake_point) -- node {$\hat{v}_{q}$} (output);
  \draw [-] (fake_point) -- node {} (fake_point2);
  \draw [->] (fake_point2) -| node {} (sum_P);
  
\end{tikzpicture}
} 
%   \end{figure}
%  Takes the estimation of the frequency deviation at the bus connection $\Delta \hat{\omega}$  and estimates the voltage at the bus $\hat{v}_q$ 
% \end{frame}

\begin{frame}[allowframebreaks]

  \printbibliography
  
\end{frame}
      
\end{document}